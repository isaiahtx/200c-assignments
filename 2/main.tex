\documentclass[10pt]{article}
\usepackage[margin=1.3333333in]{geometry}
\usepackage{amsfonts}
\usepackage[]{eucal}
\usepackage{amsmath}
\usepackage{amssymb}
\usepackage{amsthm}
\usepackage{enumitem}
\usepackage{fancyhdr}
\usepackage{mathrsfs}
\usepackage{mathtools}
\usepackage{physics}
\usepackage{tabularx}
\usepackage{tikz}
\usepackage{verbatim}
\usepackage{xcolor}
\usepackage{quiver}
\setlength{\parindent}{0pt}

\raggedbottom
\pagestyle{fancy}
\lhead{}
\chead{200C Assignment 2}
\rhead{}
\cfoot{\thepage}

\newenvironment{solution}
{\renewcommand\qedsymbol{}\begin{proof}[Solution]}
{\end{proof}}

\newcommand{\cI}{\mathcal{I}}
\newcommand{\mf}{\mathfrak}
\newcommand{\from}{\colon}
\newcommand{\sseq}{\subseteq}
\renewcommand{\k}{\mbf k}
\newcommand{\brn}{\mathbb R^n}
\newcommand{\bRn}{\mathbb R^n}
\newcommand{\bR}{\mathbb{R}}
\newcommand{\Hom}{\mathrm{Hom}}
\newcommand{\ol}{\overline}
\DeclareMathOperator*{\Frac}{\mathrm{Frac}}
\newcommand{\cO}{\mathcal O}
\newcommand{\wt}{\widetilde}
\newcommand{\Gal}{\mathrm{Gal}}
\DeclareMathOperator{\Supp}{Supp}
\DeclareMathOperator{\coker}{coker}
\DeclareMathOperator{\Ann}{Ann}
\DeclareMathOperator{\Spec}{Spec}
\DeclareMathOperator{\MaxSpec}{MaxSpec}
\newcommand{\lsim}{{{\sim}}}
\newcommand{\id}{\mathrm{id}}
\newcommand{\mbf}[1]{\mathbf{#1}}
\newcommand{\bZ}{\mathbb{Z}}
\newcommand{\bN}{\mathbb{N}}
\renewcommand{\include}{\hookrightarrow}
\newcommand{\into}{\hookrightarrow}
\newcommand{\onto}{\twoheadrightarrow}
\newcommand{\bQ}{\mathbb{Q}}
\newcommand{\bF}{\mathbb{F}}
\newcommand{\bC}{\mathbb{C}}
\renewcommand{\c}{\mathbf{c}}
\newcommand{\vare}{\varepsilon}
\newcommand{\sign}{\mathrm{sign}\,}
\newcommand{\1}{\mbf 1}
\renewcommand{\mod}{\ \mathrm{mod}\ }
\newcommand{\2}{\mbf 2}
\newcommand{\3}{\mbf 3}
\renewcommand{\v}{\mbf{v}}
\newcommand{\w}{\mbf{w}}
\renewcommand{\u}{\mbf{u}}
\newcommand{\4}{\mbf 4}
\newcommand{\5}{\mbf 5}
\newcommand{\frp}{\mathrm{Frob}_p}
\newcommand{\6}{\mbf 6}
\newcommand{\7}{\mbf 7}
\newcommand{\8}{\mbf 8}
\newcommand{\9}{\mbf 9}
\newcommand{\0}{\mbf 0}
\renewcommand{\i}{\mbf i}
\DeclareMathOperator*{\im}{\mathrm{im}}
\renewcommand{\j}{\mbf j}
\renewcommand{\k}{\mbf k}
\renewcommand{\(}{\left(}
\renewcommand{\)}{\right)}
\newcommand\defeq{\mathrel{\overset{\makebox[0pt]{\mbox{\normalfont\tiny\sffamily def}}}{=}}}
\newcommand{\phantomreplace}[2]{\makebox[0pt][l]{#1}\hphantom{#2}}
\newcommand{\phantommathreplace}[2]{\makebox[0pt][l]{$\displaystyle #1$}\hphantom{#2}}
\makeatletter
\newcommand{\skipitems}[1]{%
  \addtocounter{\@enumctr}{#1}%
}
\makeatother

\begin{document}

\begin{enumerate}[label=(\arabic*)]
    \item Let $n$ be a square-free integer (i.e., every prime divides $n$ at most once). Let $\bQ$ be the rational numbers. Define
    \[\bQ(\sqrt n)=\{a+b\sqrt n\mid a,b\in\bQ\},\]
    which is the splitting field of $x^2-n$ over $\bQ$.
    \begin{enumerate}[label=(\alph*)]
        \item If $b\neq0$, show that $a+b\sqrt n$ satisfies a unique monic degree $2$ polynomial with rational coefficients.
        \begin{proof}
            Define $f(x)=x^2-2ax+(a^2-b^2n)$. It is straightforward to verify that $f$ has $a+b\sqrt n$ as a root. Since $\bQ$ is a field, the kernel of the ring morphism $\gamma:\bQ[x]\to\bQ(\sqrt n)$ mapping $x\mapsto a+b\sqrt n$ must be principal generated by the smallest degree monic polynomial it contains, which is necessarily unique. It is not hard to see that since $b,n\neq 0$ and $n$ is square-free that $\ker\gamma$ contains no degree-$0$ or degree-$1$ polynomials, so that it must be generated by the monic polynomial $f$, which is of the next-highest degree $2$. Hence, $f$ is unique.
        \end{proof}
        \item Determine the integral closure of $\bZ$ in $\bQ(\sqrt n)$.
        
        (Hint: The answer depends on whether or not $n\equiv 1\ (\mathrm{mod}\ 4)$)
        \begin{proof}
            Fix some square-free $n\in\bN$. Let $\mathcal O_{\bQ(\sqrt n)}$ denote the integral closure of $\bZ$ in $\bQ(\sqrt n)$. By Gauss' Lemma, given $a,b\in\bQ$, we have that $a+b\sqrt n\in\cO_{\bQ(\sqrt n)}$ if and only if the minimal polynomial of $a+b\sqrt n$ in $\bQ[x]$ is an integer polynomial\footnote{Let $f\in\bZ[x]$ be a monic polynomial of minimal degree which has $a+b\sqrt n$ as a root. Since $f$ is irreducible in $\bZ[x]$, by Gauss' Lemma it is irreducible in $\bQ[x]$. Hence, $f$ is a monic irreducible polynomial which has $a+b\sqrt n$ as a root, so that $f$ must be the minimal polynomial of $a+b\sqrt n$.}. By part (a), this is furthermore true if and only if $2a,a^2-b^2n\in\bZ$. Suppose some $a,b\in\bQ$ are given satisfying this condition.
%
            %Now, let $a+b\sqrt n$ in $\bQ(\sqrt n)$ be integral over $\bZ$. By Gauss' Lemma, if $f\in\bZ[x]$ is the smallest-degree polynomial which has $a+b\sqrt n$ as a root, then $f$ is irreducible in $\bQ[x]$, so that $f$ is the minimal polynomial of $a+b\sqrt n$, which is $x^2-2ax+(a^2-b^2n)$, as we showed above. Hence, we at least know that for $a+b\sqrt n$ to be integral over $\bZ$, it must be true that $-2a,a^2-b^2n\in\bZ$. From here, we split into cases:

            \textbf{Case 1:} $a\in\bZ$. In this case, necessarily $b\in\bZ$ as well. Indeed, suppose $b=p/q$ in reduced form (so $\mathrm{gcf}(p,q)=1$). Because $p$ and $q$ share no factors, neither do $p^2$ and $q^2$. Thus, in order for $p^2n/q^2$ to be an integer, $q^2$ must be a factor of $n$, which is square-free, meaning $q=1$. Hence $b$ is an integer if $a$ is.

            \textbf{Case 2:} $a\notin\bZ$. In this case, since we know it must be true that $2a\in\bZ$, necessarily $a=m/2$ where $m$ is odd. Furthermore, it must be true that $m^2/4-b^2n\in\bZ$, which holds iff $m^2-4b^2n\in4\bZ$. 
            Write $b=p/q$ where $p$ and $q$ are coprime. Then since $m$ is odd, so is $m^2$, meaning that $4b^2n=4p^2n/q^2$ is likewise an odd integer. Then $4$ must be a factor of $q^2$, so that $q$ must be even. This further implies that $p$ must be odd, as $p$ and $q$ are coprime. Note that $4p^2n$ is divisible by no power of $2$ larger than $8$, as $p$ is odd and $n$ is square-free. Hence, $q^2\leq 8$ and $q$ is even, so $q=\pm 2$. Thus, it remains to find all $p\in\bZ$ for which $m^2-4b^2n=m^2-4p^2n/4=m^2-p^2n\in4\bZ$, i.e., those $p\in\bZ$ for which $m^2\equiv p^2n\ (\mathrm{mod}\ 4)$. Since $m$ is odd, we can write $m=2k+1$, in which case $m^2=4k^2+4k+1\equiv 1\mod 4$. Hence, any $p\in\bZ$ for which $p^2n\equiv 1\mod 4$ suffices. In particular, neither $p$ nor $n$ can be even. Furthermore, since $p$ is odd, $p^2\equiv 1\mod 4$. Hence, the only way it can be true that $p^2n\equiv 1\mod 4$ is if $n\equiv 1\mod 4$. Indeed, if $n\equiv 3\mod 4$, then we would have $p^2n\equiv 3\mod 4$.

            To recap, given $a,b\in\bQ$, if $a$ and $b$ are integers, then for any square-free $n$ $a+b\sqrt n$ is \textit{always} integral over $\bZ$. 
            
            If $n\not\equiv 1\mod 4$, then these are the only elements integral over $\bZ$, in which case $\cO_{\bQ(\sqrt n)}=\bZ[\sqrt n]$. 
            
            If $n\equiv 1\mod 4$, then $a+b\sqrt n\in\cO_{\bQ(\sqrt n)}$ if $a$ and $b$ are of the form $m/2$ and $p/2$, where $m$ and $p$ are either both even or both odd. In this case, $\cO_{\bQ(\sqrt n)}=\bZ\left[\frac{1+\sqrt n}2\right]$.
        \end{proof}
    \end{enumerate}

    \item Let $\k$ be a field and consider the two rings
    \[A=\k[x,y]/(y^2-x^3),\quad\quad B=\k[x,y]/(y^2-x^3-x^2).\]
    They are both domains (you don't have to prove this); show that in both cases the normalization is the subring of the field of fractions generated by the ring and $y/x$.

    Hint: Show that adjoining $y/x$ gives a ring which is isomorphic to a polynomial ring over $\k$ in $1$ variable.
    \begin{proof}
        First, we define an embedding $A\to\k[t]$. It suffices to define a ring morphism $\k[x,y]\to\k[t]$ with kernel $(y^2-x^3)$. Define $\varphi:\k[x,y]\to\k[t]$ to be the $\k$-linear map sending $x\mapsto t^2$ and $y\mapsto t^3$. Clearly $\ker\varphi\supseteq(y^2-x^3)$. Now, suppose that $p\in\ker\varphi$. Viewing $p$ as an element of $(k[x])[y]$, we can perform polynomial division to write $p=q(y^2-x^3)+r$, where $r$ is of degree of at most $1$ (w.r.t.\ $y$). Write
        \[r=\sum_{i=0}^n(a_ix^i+b_ix^iy).\]
        Then by additivity, $r\in\ker\varphi$, as $q(y^2-x^3)\in\ker\varphi$. Hence,
        \[\varphi(r)=\sum_{i=0}^n(a_it^{2i}+b_it^{2i+3})=0,\]
        which clearly holds if and only if $a_i=b_i=0$ for all $i$, as the $t^i$ for $i\geq 2$ are $\k$-linearly independent. In other words, $r=0$, so that indeed we have $p\in(y^2-x^3)$. Hence, $\ker\varphi=(y^2-x^3)$.
 
        Therefore, by the universal property of a quotient there exists an embedding $\wt\varphi:A\into\k[t]$ with image $\k[t^2,t^3]$. Note that $\k[t]\supseteq\k[t^2,t^3]$ is an integral extension, as $t$ is a root of the monic polynomial $z^2-t^2$ in $\k[t^2,t^3][z]$. Furthermore, note that $\Frac\k[t^2,t^3]=\Frac\k[t]$ (as $t^3/t^2=t$ belogns to $\Frac\k[t^2,t^3]$), and $\k[t]$ is a UFD, so that it is integrally closed in its field of fractions. Hence, $\k[t]$ is the normalization of $\k[t^2,t^3]\cong A$. By the universal property of the fraction field, there exists a morphism $\psi:\Frac A\to\k(t)$ sending $p/q\mapsto \wt\varphi(p)/\wt\varphi(q)$ such that the following diagram commutes
        \[\begin{tikzcd}
            & {\k[t]} & {\k(t)} \\
            A \\
            & {A[\tfrac yx]} & {\Frac A}
            \arrow["\wt\varphi", hook, from=2-1, to=1-2]
            \arrow[hook, from=1-2, to=1-3]
            \arrow[hook, from=2-1, to=3-2]
            \arrow[hook, from=3-2, to=3-3]
            \arrow["\psi"', dashed, hook, from=3-3, to=1-3]
        \end{tikzcd}\]
        It is straightforward to see that given any $f(x,y,y/x)\in A[\tfrac yx]$, that $\psi(f)=f(t^2,t^3,t)\in\k[t]$. Furthermore, given any $f(t)\in\k[t]$, we have that $f(y/x)\in A[\tfrac yx]$ maps to $f$ via $\psi$, so that $\psi|_{A[\tfrac yx]}:A[\tfrac yx]\to\k[t]$ is both injective and surjective (injective because $\psi$ is a nontrivial morphism of fields), hence, an isomorphism. It follows that $A[\tfrac yx]$ is the normalization of $A$.

        %Define a $\k$-linear ring morphism $\varphi:\k[t]\to B[\tfrac yx]$ by $t\mapsto y/x$. First, we claim that $\varphi$ is surjective. It further suffices to show that $x$ and $y$ are in the image of $\varphi$. Indeed, working modulo the relation $y^2=x^2(x+1)$, we have that
        %\[\varphi(t^2-1)=y^2/x^2-1\overset{(\ast)}=x+1-1=x,\]
        %where $(\ast)$ follows by the fact that $y^2=x^2(x+1)$ in $B[y/x]$. Similarly, we have
        %\[\varphi(t^3-t)=\varphi(t)\varphi(t^2-1)=\frac yx\cdot x=y.\]
        %Hence, indeed $\varphi$ is surjective. Now, we claim that $\varphi$ is injective.
%
        \pagebreak 
        First, we define an embedding $B\to\k[t]$. It suffices to define a ring morphism $\k[x,y]\to\k[t]$ with kernel $(y^2-x^3-x^2)$. 
        Define $\varphi:\k[x,y]\to\k[t]$ to be the $\k$-linear map sending $x\mapsto t^2-1$ and $y\mapsto t^3-t$. A routine calculation yields that $y^2-x^3-x^2\in\ker\varphi$. Now, suppose that $p\in\ker\varphi$. Vieweing $p$ as an element of $(k[x])[y]$, we can perform polynomial division to write $p=q(y^2-x^3-x^2)+r$ where $r$ is of degree at most 1 (w.r.t.\ $y$). Write
        \[r=a(x)+y\cdot b(x),\]
        where $a,b\in\k[x]$. Then by additivity, $r\in\ker\varphi$, as $q(y^2-x^3-x^2)\in\ker\varphi$. Hence,
        \[\varphi(r)=a(t^2-1)+t(t^2-1)b(t^2-1)=0.\]
        Note that $a(t^2-1)$ will necessarily be an even-degree polynomial or a constant term, while $t(t^2-1)b(t^2-1)$ will be an odd-degree polynomial. Hence, the only way for it to be true that 
        \[a(t^2-1)+t(t^2-1)b(t^2-1)=0\]
        is if $a(t^2-1)=b(t^2-1)=0$, which in turn is true if and only if $a=b=0$. Hence, $r=0$, so that $p=q(y^2-x^3-x^2)\in(y^2-x^3-x^2)$. Hence $\ker\varphi=(y^2-x^3-x^2)$. 

        Therefore, by the universal property of a quotient there exists an embedding $\wt\varphi: A\into\k[t]$ with image $\k[t^2-1,t(t^2-1)]$. Note that $\k[t]\supseteq\k[t^2-1,t(t^2-1)]$ is an integral extension, as $t$ is a root of the monic polynomial $z^2-(t^2-1)-1\in \k[t^2-1,t(t^2-1)][z]$. Furthermore, note that $\Frac\k[t^2-1,t(t^2-1)]=\Frac\k[t]$ (as $t(t^2-1)/(t^2-1)=t$ belongs to $\Frac\k[t^2-1,t(t^2-1)]$), and $\k[t]$ is a UFD, so that it is integrally closed in its field of fractions. Hence, $\k[t]$ is the normalization of $\k[t^2-1,t(t^2-1)]\cong A$. By the universal properry of the fraction field, there exists a morphism $\psi:\Frac A\to\k(t)$ sending $p/q\mapsto\wt\varphi(p)/\wt\varphi(q)$ such that the following diagram commutes.
        \[\begin{tikzcd}
            & {\k[t]} & {\k(t)} \\
            A \\
            & {A[\tfrac yx]} & {\Frac A}
            \arrow["\wt\varphi", hook, from=2-1, to=1-2]
            \arrow[hook, from=1-2, to=1-3]
            \arrow[hook, from=2-1, to=3-2]
            \arrow[hook, from=3-2, to=3-3]
            \arrow["\psi"', dashed, hook, from=3-3, to=1-3]
        \end{tikzcd}\]
        It is straightforward to see that given any $f(x,y,y/x)\in A[\tfrac yx]$, that $\psi(f)=f(t^2-1,t(t^2-1),t)\in\k[t]$. Furthermore, given any $f(t)\in\k[t]$, we have that $f(y/x)\in A[\tfrac yx]$ maps to $f$ via $\psi$, so that $\psi|_{A[\tfrac yx]}:A[\tfrac yx]\to\k[t]$ is both injective and surjective (injective because $\psi$ is a nontrivial morphism of fields), hence, an isomorphism. It follows that $A[\tfrac yx]$ is the normalization of $A$.
    \end{proof}

    \item \begin{enumerate}[label=(\alph*)]
        \item Let $A$ be a ring and $f=t^n+a_1t^{n-1}+\cdots+a_n$ be any monic polynomial with coefficients in $A$. Define the \textbf{splitting ring} $S_A(f)$ of $f$ to be
        \[S_A(f)=A[\xi_1,\ldots,\xi_n]/I\]
        where $\xi_1,\ldots,\xi_n$ are variables, and $I$ is generated by the coefficients of
        \[(t-\xi_1)\cdots(t-\xi_n)-f(t)\]
        thought of as a polynomial in $t$. Show that the natural map $A\to S_A(f)$ is integral (you don't need to prove it is injective, though that is true).
        \begin{proof}
            It suffices to show that each $\xi_i$ is integral over $A$ for $i=1,\ldots,n$. Note that $f$ is a monic polynomial with coefficients in $A$, so it further suffices to show that $f(\xi_i)=0$. 

            For $j=1,\ldots,n$, let $e_j$ be the coefficient of the $(n-j)^\text{th}$ term of $(t-\xi_1)\cdots(t-\xi_n)$. Then the generators of $I$ are the elements $e_j-a_j$ for $j=1,\ldots,n$, so that working modulo $I$,
            \[f(\xi_i)=\xi_i^n+a_1\xi_i^{n-1}+\cdots+a_n=\xi_i^n+e_1\xi_i^{n-1}+\cdots+e_n=(\xi_i-\xi_1)\cdots(\xi_i-\xi_n)=0,\]
            so that indeed $\xi_i$ is integral over $A$, $f$ is integral.
        \end{proof}
        \item (Atiyah-Macdonald, Exercise 5.8.ii). Let $A$ be a subring of $B$, and let $C$ be the integral closure of $A$ in $B$. Let $f,g$ be monic polynomials in $B[x]$ such that $fg\in C[x]$. Then $f,g$ are in $C[x]$.
        \begin{proof}
            Let $\deg f=n$ and $\deg g=m$. We start by constructing a ring $B'$ over which $f$ and $g$ split completely into linear factors. Define $B_1$ to be the ring $B[t_1]/(f(t_1))$. Viewed as an element of $B_1[x]$, $f(x)$ has a root $\ol{t_1}$ in $B_1$. Furthermore, $f(x)$ is in the kernel of the quotient map $B_1[x]\onto B_1[x]/(x-\ol{t_1})$, so that we can write $f(x)=(x-\ol{t_1})f_1(x)$ for some polynomial $f_1(x)\in B_1[x]$. In particular, note that $\deg f_1=n-1$. We can then construct $B_2$ to be the quotient ring $B_1[t_2]/(f_1(t_2))$, adjoining another root of $f$. Again $f_1(x)$ is clearly in the kernel of the quotient map $B_2[x]\onto B_2[x]/(x-\ol{t_2})$, so that there exists a polynomial $f_2(x)\in B_2[x]$ of degree $n-2$ with $f_1(x)=(x-\ol{t_2})f_2(x)$. We can proceed in this manner until we have constructed $B_n$, in which $f$ splits completely as $(x-\ol{t_1})(x-\ol{t_2})\cdots(x-\ol{t_n})$. In a similar manner, we can adjoin the roots of $g$ to $B_n$ one-by-one until we have obtained a ring $B'$ over which both $f$ and $g$ split entirely into linear factors, say as
            \[f=\Pi(x-\xi_i)\quad\quad\text{ and }\quad\quad g=\Pi(x-\eta_j).\]
            Each $\xi_i$ and $\eta_j$ is a root of $fg$ and therefore is integral over $C$. Hence the coefficients of $f$ and $g$, which are polynomials in the $\xi_i$'s and $\eta_j$'s respectively, are also integral over $C$, and therefore belong to $C$. Thus $f,g\in C[x]$.
        \end{proof}
    \end{enumerate}

    \item (Atiyah-Macdonald, Exercise 5.12). Let $G$ be a finite group of automorphisms of a ring $A$, and let $A^G$ denote the subring of $G$-invariants, that is of all $x\in A$ such that $\sigma(x)=x$ for all $\sigma\in G$. Prove that $A$ is integral over $A^G$.
    \begin{proof}
        First, we show that $A^G$ is a ring. Given $a,b\in A^G$ and $\sigma\in G$, we have by the fact that $\sigma$ is a ring morphism that
        \[\sigma(1)=1,\quad\quad\quad\sigma(ab)=\sigma(a)\sigma(b)=ab\quad\quad\text{and}\quad\quad\sigma(a-b)=\sigma(a)-\sigma(b)=a-b,\]
        so that indeed $1,ab,a\pm b\in A^G$.

        Let $a\in A$ and define $p:=\prod_{\sigma\in G}(x-\sigma(a))\in A[x]$. First, we claim that $p\in A^G[x]$. It suffices to show that $\tau(p)=p$ for all $\tau\in G$ (where we implicitly extend $\tau:A\to A$ to a map $A[x]\to A[x]$ simply sending $x\mapsto x$). Indeed, we have:
        \[\tau(p)=\tau\(\prod_{\sigma\in G}(x-\sigma(a))\)=\prod_{\sigma\in G}(x-\tau(\sigma(a)))\overset{(\ast)}=\prod_{\sigma\in G}(x-\sigma(a)),\]
        where $(\ast)$ follows by the fact that the group homomorphism $G\to G$ given by $\sigma\mapsto \tau\circ\sigma$ is an automorphism (as it has an inverse given by composition with $\tau^{-1}$). Finally, clearly $a$ is a root of $p$, as since $G$ is a group it contains the identity automorphism $\id_A:A\to A$, so that if $G=\{\id_A,\sigma_1,\ldots,\sigma_n\}$, then
        \[p(a)=(a-a)(a-\sigma_1(a))\cdots(a-\sigma_n(a))=0.\]
        Therefore $a$ is indeed integral over $A^G$.
    \end{proof}
    Let $S$ be a multiplicatively closed subset of $A$ such that $\sigma(S)\sseq S$ for all $\sigma\in G$, and let $S^G=S\cap A^G$. Show that the action of $G$ on $A$ extends to an action on $S^{-1}A$, and that $(S^G)^{-1}A^G\cong (S^{-1}A)^G$.
    \begin{proof}
        Define an action of $G$ on $S^{-1}A$ by
        \begin{align*}
            G\times S^{-1}A&\to S^{-1}A \\
            (\sigma,a/s)&\mapsto \sigma(a)/\sigma(s).
        \end{align*}
        Note that $\sigma(a)/\sigma(s)$ is indeed a valid element of $S^{-1}A$ as $\sigma(S)\sseq S$.
        First, we show that this is well-defined. Suppose $a/s=b/t$ in $S^{-1}A$, so that there exists $x\in S$ such that
        \[x(ta-sb)=0.\]
        Then
        \[\sigma(x)(\sigma(t)\sigma(a)-\sigma(s)\sigma(b))=\sigma(x(ta-sb))=\sigma(0)=0,\]
        so that $\sigma(a)/\sigma(s)=\sigma(b)/\sigma(t)$ via the element $\sigma(x)\in S$. We further claim that each $\sigma\in G$ acts as a ring endomorphism on $S^{-1}A$. Indeed, it is multiplicative:
        \[\sigma\(\frac as\cdot\frac bt\)=\sigma\(\frac{ab}{st}\)=\frac{\sigma(ab)}{\sigma(st)}=\frac{\sigma(a)\sigma(b)}{\sigma(s)\sigma(t)}=\frac{\sigma(a)}{\sigma(s)}\cdot\frac{\sigma(b)}{\sigma(t)}=\sigma\(\frac as\)\cdot\sigma\(\frac bt\),\]
        and additive:
        \[\sigma\(\frac as+\frac bt\)=\sigma\(\frac{ta+sb}{st}\)=\frac{\sigma(ta+sb)}{\sigma(st)}=\frac{\sigma(t)\sigma(a)+\sigma(s)\sigma(b)}{\sigma(s)\sigma(t)}=\frac{\sigma(a)}{\sigma(s)}+\frac{\sigma(b)}{\sigma(t)}=\sigma\(\frac as\)+\sigma\(\frac bt\).\]
        
        Now, I claim $(S^G)^{-1}A^G\cong(S^{-1}A)^G$. Define a ring morphism $\varphi:A^G\to (S^{-1}A)^G$ by $a\mapsto a/1$. Note that indeed if $a\in A^G$, then $a/1\in(S^{-1}A)^G$, as for all $\sigma\in G$ we have $\sigma(a/1)=\sigma(a)/\sigma(1)=a/1$. It is not hard to verify that this is a ring morphism:
        \[\varphi(1)=\frac11\quad\quad\text{and}\quad\quad\varphi(a+bc)=\frac{a+bc}1=\frac a1+\frac b1\cdot\frac c1=\varphi(a)+\varphi(b)\varphi(c).\]
        Furthermore, $\varphi$ sends every element in $S^G$ to a unit in $(S^{-1}A)^G$, as given $s\in S^G$ we have
        \[\varphi(s)\cdot\frac1s=\frac s1\cdot\frac1s=\frac ss=\frac 11.\]
        Hence, by the universal property of localization, there exists a morphism $\wt\varphi:(S^G)^{-1}A^G\to(S^{-1}A)^G$ sending $a/s\mapsto\varphi(a)\varphi(s)^{-1}=(a/1)(1/s)=a/s$. We claim $\wt\varphi$ is an isomorphism. 
        
        First, we show that it is injective. Let $a/s\in(S^G)^{-1}A^G$ such that $\wt\varphi(a/s)=a/s$ is zero in $(S^{-1}A)^G$, so that there exists $t\in S$ with $ta=0$. Define
        \[t':=\prod_{\sigma\in G}\sigma(t),\]
        then $t'a=0$ as well, so that $a/s\in(S^G)^{-1}A^G$.

        Finally, we claim that $\wt\varphi$ is surjective. Let $a/s\in (S^{-1}A)^G$ so that there exists $t\in S$ such that $ts\sigma(a)=t\sigma(s)a$. Set $t'=\prod_{\sigma\in G}\sigma(t)$. Define $a'$ and $s'$ similarly. Then
    \end{proof}
    
    \item (Atiyah-Macdonald, Exercise 5.9). Let $A$ be a subring of a ring $B$ and let $C$ be the integral closure of $A$ in $B$. Prove that $C[x]$ is the integral closure of $A[x]$ in $B[x]$. 
    \begin{proof}
        First, we show that $C[x]\supseteq A[x]$ is an integral extension. Let $f\in C[x]$. By Proposition 3.1.1, it suffices to show that there exists a subring $C'$ of $B[x]$ that contains $A$ and $f$ such that $C'$ is a finitely generated $A[x]$-module. Set $C':=C[x]$. Since $C$ is a finitely generated $A$ module, clearly $C[x]$ is a finitely generated $A[x]$ module, giving the desired result.

        Secondly, we show that $C[x]$ is integrally closed in $B[x]$. Suppose $f\in B[x]$ is integral over $C[x]$, so that $-f$ is also integral over $C[x]$, meaning there exists $g_1,\ldots,g_n\in C[x]$ such that
        \[(-f)^n+g_1(-f)^{n-1}+\cdots+g_{n-1}(-f)+g_n=0.\]
        Then let $r$ be an integer greater than the degree of $f$ and each $g_i$, and let $f_1$ be the monic polynomial $x^r-f$. Then
        \[(f_1-x^r)^n+g_1(f_1-x^r)^{n-1}+\cdots+g_{n-1}(f_1-x^r)+g_n=0.\]
        Expanding, we have that there exists $h_1,\ldots,h_n\in C[x]$ such that
        \[f_1^n+h_1f_1^{n-1}+\cdots+h_{n-1}f_1+h_n=0,\]
        where in particular 
        \[h_n=g_n+g_{n-1}x^r+g_{n-2}x^{2r}+\cdots+g_2x^{r(n-2)}+g_1x^{r(n-1)}\in C[x].\]
        Then
        \[f_1^n+h_1f_1^{n-1}+\cdots+h_{n-1}f_1=-h_n\in C[x],\]
        so that
        \[f_1(f_1^{n-1}+h_1f_1^{n-2}+\cdots+h_{n-2}f_1+h_{n-1})\in C[x],\]
        so that by Question 3(b), $f_1\in C[x]$, meaning $f=x^r-(x^r-f)\in C[x]$ by Corollary 3.1.2.
    \end{proof}
\end{enumerate}
\end{document}